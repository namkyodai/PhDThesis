\begin{center}
   % \setlength{\parskip}{0pt}
    {\huge{Executive Summary} \par}
 %  \bigskip
    \end{center}
  \vfil\vfil
Recent studies on infrastructure asset management pay great attention on formulating the best fitting stochastic hazard models and on the solutions to problems arising from incomplete monitoring data. Hazard model plays a center role in any infrastructure management system because of its ability to predict the deterioration. Meanwhile, monitoring data is the primary sort of information, which is necessary to be used in the empirical application of a hazard model. The development of hazard models requires  understanding of the deterioration mechanism/process on the entire operational life cycle and the dependence of deterioration on characteristic variables. In hazard analysis with stochastic approach, deterioration mechanism can be simulated by means of transition among  discrete condition states (healthy status of infrastructure system), which are quantified by aggregate values of distress indexes recorded via regular monitoring and visual observation.

Evidently, the deterioration process or transition among condition states depends on the changes in values of characteristic variables over a period. For example, the cracking of pavement progresses in close link with the increasing or decreasing of traffic volume, thickness of overlay structure, and ambient temperature. To understand the deterioration mechanism and the dependencies on characteristic variables, monitoring, and visual inspection are indispensable in management of any infrastructure system. However, there is a fact that continuous monitoring and inspection are often technically and financially difficult. As a sequent, monitoring data is generally incomplete. Thus, in formulating  hazard models and in monitoring characteristic variables, it is important to define a suitable deterioration mechanism along with a good selection of characteristic variables for particular infrastructure system.

A great deal of past researches paid much attention to the physical mechanism of deterioration of structures. However, the past research remained in a rudiment stage of development as not specifying a clear statistical estimation method. Thus, several problems from the estimation results can be seen as the limitations. 

In stochastic hazard models, the application of Markov chain model has become popular. Markov chain model has its advantage that it requires only monitoring data of two visual inspection times. Thus, it reduces the burdens of collecting continuous monitoring data and full-scale inspection. However, the estimation of hazard rate and transition probability matrix in Markov chain model is not an easy task. Especially when having to tackle the problems of multi-condition states, system with memory, measurement errors, and inhomogeneous monitoring data.

Among prominent problems, the assumption of condition states and analytical estimation method in hazard analysis are often discussed. In reality, the deterioration of most of infrastructure systems should be expressed in multi-condition states. However, due to either poor definition or lack of monitoring technique, binary condition state regime is applied instead. This over simplification prevents managers/engineers from selecting choices for maintenance and repair. The multi-condition state regime, on the other hand, requires advanced monitoring technologies and sophisticated calculation. Thus, selection of binary condition state regime or multi- condition state regime crucially depends on the availability of monitoring data and on the requirements of maintenance and repair. Another popular problem in monitoring data is measurement errors. The systematic errors occur and exist in the database system due to either defection of monitoring devices or human mistakes. These measurement errors, if used in hazard models, will bias the estimation results. 

There is another important issue in management of infrastructure, especially for underground infrastructure system, where monitoring techniques exerts to require huge cost and time. Moreover, critical damage or failure of system often generates huge loss in social and repair costs. Thus, finding the optimal renewal time for such system is crucial important. A great number of studies have proposed models with aims for optimal renewal time. However, most of them used non-homogeneous Poisson process, which did not take into consideration of the in-service duration of structure. 

The last problem is discussed in this research is the methodology to estimate heterogeneity factor in mixture hazard model, which is used for inhomogeneous set of monitoring data. The estimation methodology for mixture hazard model has not been precisely established in the field since the difficulty is on the assumption of heterogeneity factor to follow parametric behavior or a function. The study on mixture hazard model will enable the study of benchmarking, which is used to find the best practice in management and technology. In view of pavement management in developing countries, where many different borrowed technologies are applied, finding a best technology would bring in significant results.

The study aimed at formulating stochastic optimization methods for infrastructure asset management under incomplete monitoring data. The objectives and scopes of the study were organized into two main parts. The first part presents two innovative hazard models attempting to promote the application of multi-condition states regime by use of Weibull hazard functions, and to solve the problems of measurement errors in monitoring data by employing hidden Markov model with Bayesian and Markov Chain Monte Carlo (MCMC) methodology. The second part aimed at development of hazard models dealing with optimal renewal time, life cycle cost estimation, and benchmarking based on the core of hazard models in the first part. Empirical studies of the developed hazard models and methodologies were conducted on incomplete monitoring data of four main infrastructure systems: The lighting utility in tunnel system in Japan, the express highway system in Japan, the water distribution pipelines system in Japan, and the pavement management system in Vietnam.

In \textbf{Chapter \ref{Chapter3}}, a time-dependent deterioration-forecasting model was presented, whereby the deterioration is described by the transition probabilities, which conditionally depends on the actual in-service duration. We formulated the model by use of multi-stage Weibull hazard functions. The study had solved the critiques over the hazard model with binary condition state regime. Moreover, by employing Weibull functions for representing the behavior of hazard rate, the study further addressed the importance of monitoring data, which should also capture the historical performances of infrastructure as sufficient as possible. The model can be estimated based upon the incomplete monitoring data, which are obtained at the discrete points in time. The applicability of the model and the estimation methodology presented in this chapter was investigated with empirical study on $12,311$ data samples of the highway tunnel lighting utilities in Japan.

For tunnel lighting utility as a case, the range of condition states were defined in the domain [1-4] for ease of monitoring and maintenance. The overall life expectancy of both normal lighting and ease lighting utilities was about $13$ years. Interestingly, empirical study revealed that the results obtained by using the Multi-stage Weibull hazard had been significantly improved if comparing with the results produced by using the conventional Markov model. The conventional Markov model is the model with hazard function to follow exponential form, which was briefly introduced in the literature of \textbf{Chapter \ref{Chapter2}}. The differences of overall life expectancy estimated by two models were about $3$ years to $4$ years. The longer life expectancy produced by applying conventional Markov model can be claimed to incomplete monitoring data, data without censoring, and the computation using only two most recent sampling populations. Based on the distribution of condition states over the years, it is advisable for tunnel administrator to carry out  inspection after $5$ years to $6$ years from the opening of services.

Measurement errors in monitoring data were extensively discussed in \textbf{Chapter \ref{Chapter4}}. As earlier mentioned, the problem of measurement errors in monitoring data  tends to bias the estimation results of the conventional Markov hazard model. As a matter of course, measurement errors can be more or less eliminated by using some simple sorting techniques such as: correcting or erasing samples with better condition states in the second observation than the first observation. However, sorting techniques cannot reveal latent errors. To uncover and solve the problems, a hidden Markov model was formulated and presented in this chapter. In the hidden Markov model, measurement errors are assumed as random variables. The estimation methodology was developed with aids of Bayesian estimation and MCMC technique in tackling the posterior probability distribution and sampling generation of condition states. An empirical application on Japanese national road system was presented to demonstrate the applicability of the model. The estimation results highlighted that the properties of Markov transition matrix had been greatly improved in comparison with the properties obtained from using the conventional exponential hazard model.

In the empirical study of the hidden Markov model, we used $5,261$ numbers of samples of Japanese expressways collected during the period from $1998$ to $2005$. Each sample represented $100$ meters of expressway. The healthy status of sections were evaluated by means of $5$ discrete condition states, with $1$ as the best condition state and $5$ as the worst condition state. The condition states were converted values based on the range of  rut index . Estimation results showed the fact that measurement errors had existed in the monitoring data for a long time. Measurement errors caused the deterioration curve, which was estimated by applying the exponential hazard model, to sharply decrease in comparison with the true deterioration curve. In addition, by applying the hidden Markov model, it was possible to have a re-produced database, which yielded the results closely to the true values. The overall life expectancy of overlay structure of the Japanese expressway was predicted to be about $30$ years and $35$ years. 

In \textbf{Chapter \ref{Chapter5}}, we discussed the formulation of a time-dependent hazard model using for finding optimal renewal time of underground infrastructure, where monitoring and visual observation require special techniques and huge cost. In addition, social cost and direct cost for maintenance or repair are extremely high in comparison with other structures like pavement and bridge. We considered underground water pipelines system as an example for empirical study. Underground water supply pipelines system often exerts to have high uncertainty of being leaked after several decades of operation due to the corrosion process that is not easily observed. The leakage of pipelines visually appears without early notices and requires an immediate renewal. Thus, determining an optimal time for renewal is always of essence in practice. This chapter presented a mathematical model using to define optimal renewal timing with respect to optimal total life cycle cost (LCC). In the model, the deterioration of pipelines system was formulated by employing Weibull hazard function. In view of long-term management plan, the model can be used to define the best pipeline technologies, the switching rates, and switching cost in the situation of having technology innovation.

We implemented an empirical study for the model of \textbf{Chapter \ref{Chapter5}} on the monitoring data of underground water distribution system of Osaka city. The water pipelines system included four types of pipelines according their material differences. The old fashion types of pipelines were made of cast-iron and constructed about $30$ years to half century ago, and the ductile cast-iron were newly introduced into the system about a decade ago. Estimation results showed that relatively after $70$ years from the construction time, the survival probabilities of old fashion types of pipelines become more than $0.5$. Meanwhile, it takes about a century for the survival probability of ductile cast-iron to reach to that level. Given the fixed amount of social cost, direct repair cost, and discount factor of $0.04$, the switching rates by replacing old cast-iron pipelines with innovative ductile cast-iron were defined. In order to demonstrate the effects of social cost, direct renewal cost, and the discount factor on the switching rates, we proposed a methodology using sensitivity analysis, which provided a comparative pictures for selection of the best managerial choice.

\textbf{Chapter \ref{Chapter6}} presented a mixture hazard models with Markov chain model in its core. Mixture hazard model was introduced to solve the problem of inhomogeneous monitoring data. In mixture model, the entire monitoring data is viewed as a collection of sub-sampling populations or groups of infrastructure components sharing similar characteristics, structural functions, and under same environmental conditions. In order to estimate the deterioration of an individual group $k$ in total of $K$ groups, the hazard function was defined as multiplicative form of hazard rate $\theta^k$ and heterogeneity factor $\epsilon^k$. Heterogeneity factors can follow either parametric distribution with Gamma functions or semi-parametric functions with the expansion of Taylor series. To estimate the value of heterogeneity factor, two-steps estimation approach was proposed.

Mixture hazard model is considered as an excellent tool for benchmarking implementation. By applying empirical study on targeted infrastructure groups, it is possible to propose the best group of infrastructure in term of performance and least life cycle cost. In view of long-term infrastructure management for developing countries, where exists many different technologies borrowing from developed countries, this chapter recommended the implementation of mixture model and benchmarking approach to find out the best infrastructure technology, particularly for the pavement management system.

We conducted an empirical study for mixture model and benchmarking implementation on a dataset of Vietnamese national highway collected during the period from $2001$ to $2004$. The healthy status of highway sections were classified in $5$ discrete scales, with $1$ as the best condition state and $5$ as the worse condition states. Condition states were converted values based on the range of international roughness index (IRI). In total, there were $6510$ highway sections using for the empirical test (each section is equivalent to the length of $1$ km). The characteristic variables were traffic volume and texture depth. Estimation results showed that the average life expectancy of highway sections is about $13$ years after the opening of services. Traffic volume was among the main factors causing the fast deterioration speed, especially on condition state $4$. To benchmark for the best type of highway materials, we divided the data set into $3$ main groups (Bituminous penetrated macadam, bituminous surface treatment, and asphalt concrete). It is evaluated from the mixture model that the life expectancy of the asphalt concrete highway was about $16$ years, $7$ to $8$ years longer than the life expectancy of the other two types of materials. We further categorized the asphalt concrete highway into $7$ sub-groups and used a simple cost evaluation technique to find out the sub-group, which yielded the least life cycle cost. We carried out benchmarking study also for $6$ other groups according to the distribution in geographical conditions and climate zones. With this group, it was found that the deterioration speeds of highways in the Southern regions of the country were faster than that of the highways in the Northern and Centre regions. The differences varied relatively from $4$ to $8$ years. The results provided a good picture on deterioration status and comparative life cycle costs of highways in Vietnam.
