% Chapter 7

\chapter{Conclusions} % Write in your own chapter title
\label{Chapter7}
\lhead{Chapter 7. \emph{Conclusions}} % Write in your own chapter title to set the page header
\section{A Brief Summary}
This dissertation has presented two major research directions in the field of infrastructure asset management, the development of innovative mathematical models based on Markov chain theory and the development of methodologies based on the derivatives of hazard models for optimization of infrastructure system. The development of models in the first direction provided a solid background for the second direction in extending and applying hazard models to solve the problems in the real situations. 

In the first direction, the study encompassed the formation of two innovative mathematical models, which are used mainly for forecasting purposes. The first model deals with the system of multi-condition states, where the hazard rate is subjected to be influenced by the entire historical data. To cope with this requirement, we introduced the Weibull hazard function as a basic characteristic function of deterioration process. The second model addressed the measurement errors in database system and further proposed a hidden Markov method to eliminate the errors so as to produce a closer deterioration curve to the real one. In hidden Markov model, we introduced the application of Bayesian updating rule and Markov Chain Monte Carlo, which are believed to greatly contribute to the academic researches in the field.

The second direction targeted mainly on development of methodologies to apply derivative hazard models, which have already been extensively discussed in literature review and in the first direction, for management purposes. There were two extended hazard models being proposed. The first hazard model was the optimal renewal timing model based on least life cycle cost evaluation technique, which was found to be applicable to underground infrastructure system. The second model was mixture hazard model, which is regarded as an important derivative of Markov chain model, is applied within the framework of benchmarking application.

Each presented models were then tested through the empirical application on the database of targeted infrastructure systems such as: tunnel facility, water pipeline distribution network and pavement system in either Japan or Vietnam. Details of the problems, motivations for formulation of models and results of empirical studies have been given in respective chapters of this dissertation. Despite the differences in the titles of chapters and empirical applications, all four presented models from Chapter \ref{Chapter3} to Chapter \ref{Chapter6} exhibit a close link to each other in term of stochastic estimation. Evidently, the link can be easily recognized through the definition and description of hazard rate, Weibull or exponential hazard functions and the conventional Markov chain model.
\section{Conclusions}
In a nutshell, some brief concluding points are highlighted as follows
\begin{itemize}
 \item Stochastic models using the Markov chain can give optimal solutions under the managerial requirements for infrastructure management at network level. Derivative models in this streamline could be applicable for various types of infrastructure systems beside pavement management system or tunnel lighting utilities.
 \item The model with multi-stage Weibull hazard functions can greatly improve the quality of deterioration forecasting in comparison with conventional Markov chain model. Precisely under the circumstance that the entire historical performance of infrastructure is considered (Chapter \ref{Chapter3}).
  \item Measurement errors, which are often embedded in the infrastructure inventory system, can be eliminated if hidden Markov model is applied (Chapter \ref{Chapter4}). 
  \item For underground infrastructure facilities, management finds its possibility to actually implement optimal renewal scheme based on stochastic forecasting model and life cycle cost analysis. Furthermore, a switching rate between technologies, either by means of time or cost, can be possibly visualized under the course of technology innovation (Chapter \ref{Chapter5}). 
  \item Mixture hazard model is suitable methodology for estimation of heterogeneity factors, which characterizes the inhomogeneous attributes of infrastructure system comprised of many sub or branch categories. Especially, mixture model has been proved to be a suitable managerial tool for benchmarking study in developing countries (Chapter \ref{Chapter6}). 
\end{itemize}%\newpage 
%\section{Recommendations}
%Recommendations are preferred to be in a precise way, item by item, which must be grounded on problem findings and specific conclusions. Thus, it is suggested for the readers to refer to summary section of each chapter.

