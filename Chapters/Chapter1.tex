% Chapter 1

\chapter{Introduction} % Write in your own chapter title
\label{Chapter1}
\lhead{Chapter 1. \emph{Introduction}} % Write in your own chapter title to set the page header
%%%
\section{General introduction}
Infrastructure asset management is a newly established field of research in recent years. It is now attracting a great attention from researchers and decision makers  in either developed nations or developing nations. In developed nations, there is a strong need to build up advance systemic asset management system in order to uphold the smooth operation of mass construction works built during the economic boom period half century ago. Whilst, in developing nations, due to shortage of resources, they are urging to assemble a suitable technology and program aiming toward sustainable development and meeting the speedy demand of economic growth. Regardless of the differences in demand among nations and systems, the radical discipline of infrastructure asset management is analogous as we can perceive the concepts from its wide range of the definitions \cite{kcleong}.

The entire infrastructure network of a modern society encompasses different systems specifying by their own characteristics and distinguishing management approaches. It is therefore a unique definition might not cover all the aspects. Nevertheless, according to \citet{kobaasset}, we can understand a broaden definition of infrastructure asset management as ``the optimal allocation of the scare budget between the new arrangement of infrastructure and rehabilitation/maintenance of the existing infrastructure to maximize the value of the stock of infrastructure and to realize the maximum outcomes for the citizens''.
%
%The simple extended idea on managerial principle of infrastructure asset management is straightforward from the definition, interestingly and closely, we can infer to a meaningful story ``the goose and the golden egg''. In the story, the farmer desperately could not stop in wanting for the golden egg, and with loss of his shallow enthusiasm, he finally killed the goose. And thus, in the end of the day, he lost everything, no more golden egg. In this aspect, when the entire infrastructure system is seen as a giant public asset ``A golden egg'', if our citizen only utilizes and abandons the maintenance works aside, very soon after a few years, when the deterioration rapidly reduce its serviceability, many adverse impacts resulting in the downgrade of social economic will face its destiny.

Straight from the definition, it is important to raise several critical questions like ``How can we propose optimal allocation of budget?, Which maintenance/repair strategy is the most suitable one for long-term infrastructure management? and ``Which methodology should we use to maximize the value of infrastructure stocks?''. Attempting to answer these questions have been accumulating in the research of stochastic hazard analysis, reliability study, and optimization research. In stochastic hazard analysis, researchers have been trying to formulate hazard models, which can be used to predict the deterioration process of particular infrastructure system. In addition, by employing the cost evaluation techniques and methodologies in operation researches, the stochastic hazard models can be extended to incorporate cost evaluation techniques. As a result, optimal allocation of budget and best maintenance/repair strategy can be reached.

Stochastic hazard models for optimization of infrastructure management have been widely documented in academic research \cite{madanat95,Mishalani02,jido,ziad}. Recent studies focus much on formulation and application of Markov hazard models \cite{Takeyama,kobayashitsuda,Morcous05}, particularly in pavement management system (PMS) and bridge management system (BMS) \cite{pontis,Robelin07,kumada}. As a matter of course, the formulation and application of hazard model depends largely on the mechanism of structural deterioration and monitoring data with respect to specific infrastructure systems. One  infrastructure system or component has its structural deterioration mechanism differently from that of others because of the differences in structural characteristics and in-service environment status. These differences also generate technical difficulties, time, and resource limitation for monitoring activities. It is therefore important to focus on formulating optimization methods, and to implement the methods on actual infrastructure management.
%
%In the fast changing society like today, the rate of deterioration and obsolesces is ever greater than before. However, the managerial tools and methodologies are not yet in well position to respond in an appropriate manner. Furthermore, allocation of resources on infrastructure asset management often faces great challenges due to the scarcity \cite{asko06}. Thus, there is a pressing need to develop a storehouse of knowledge, methodologies and hazard models in this newly field.
%
\section{Problem Statement}
%
In the field of infrastructure management, a great number of hazard models on deterioration forecasting have been widely documented. One major feature of the hazard models is its ability to simulate the deterioration of an infrastructure system. Beside, the hazard models can be utilized for setting up the maintenance and repair strategies as well as proposing life cycle cost analysis. Especially, under requirement of infrastructure management at network level, these objectives are particular imperative \citep{aokia,kobayashitsuda}.

A great deal of past research had paid attention to the physical mechanism of deterioration of structures \cite{mishalani95,steven}.However, the past research remained in rudiment stage of development as not specifying a clear statistical estimation method. Thus, several problems from the estimation results can be seen as the limitations. Moreover, a great number of monitoring data are generally required to ensure the accuracy of the estimation.

In recent decades, the research on statistical application have been extensively recorded   \cite{lancaster90,gouri}. For instance, \citet{shin} proposed a Weibull deterioration hazard model to forecast the starting time of crack on pavement structures. In similar approach, \citet{aokia} empirically verified the effectiveness of applying the Weibull distribution function to forecast the deterioration of tunnel lighting facilities. However, as earlier mentioned, these models portrayed the deterioration progress only by using binary condition state, and thus, did not totally reflect the actual plural condition states commonly applied in the infrastructure management system.

Attemps to tackle the emerging problems had been accumulated. A typical example is the  multi-stage model developed by \citet{lancaster90} for the behavior of labor transition. In the model, he described a rational approach to estimate the transition probability from multiple condition states. The mechanism in multi-stage model is that condition states  changes from one state to other states only in one-step. This boundary limits its application into infrastructure management since the transition of condition states is often observed in more than one-steps changes. In an effort to overcome this limitation, \citet{kobayashitsuda} described the vertical transitive relation between condition states, and proposed a method to estimate Markov transition probability according to multi-stage hazard model for bridge management.

The Markov hazard model proposed by \citet{kobayashitsuda} has a wide range of application in various infrastructure systems. However, the Markov transition probability has a characteristic that the deterioration does not depend on the past deterioration. Additionally, there is no concrete guarantee that the deterioration genuinely satisfies the Markov characters. Especially, in the case when the total operation duration of infrastructure is taken into estimation. %This appealing has generated a motivation for development of this paper, which considers either multi-state transition between condition states and historical operation time.

Application of Markov hazard models requires monitoring data from at least two inspection times. Thus, the accuracy of estimation largely depends on the quality of monitoring data. Errors exist in monitoring data are referred as measurement errors arising from measurement system or inspector (human or machine), inspected objects, or from data processing and data interpretation \citep{humplick}. Measurement errors tend to cause estimation results to be different from what they should be, especially under a small pool of monitoring data.

Methods of tackling emerging problems have been proposed with focus on formulating evaluation techniques for quantifying the error term  \citep{cochran,grubbs,humplick}. In addition, to cope with small sampling population of monitoring data and measurement errors, researchers proposed estimation methodologies using Bayesian estimation technique \citep{fenghong,ben-akiva93,ben-akiva95}. In search for overcoming the problems in spatial sampling, which is viewed as an additional reason causing measurement errors, \citet{mishalanigong} proposed an optimization model using the latent Markov decision process to selecting the best sample size. However, the past research had not recommended a clear analytical method for the prediction of deterioration under measurement errors.

Hidden Markov hazard model is a branch of Markov chain model, using to eliminate measurement errors, bias, and noise of monitoring data in a system. Some of its earlier applications can be found in the study area like image processing and applied statistics \citep{robe,mac}, in which main focuses were on the accumulation of discrete-value in time series. As for infrastructure management, the application of hidden Markov hazard model has not been seen in numerous documents, but relatively only in a small scale. Most of the past research did not clearly specify the statistical estimation method being used for analysis. Thus, it remained at an early stage of development.

Several profound literatures on hidden Markov chain models can be found in the research on economic and financial engineering. These researches tried to simulate and evaluate the business cycle by using non-stationary series of information \citep{hami89}. One of the important findings was that the confrontational change of longitudinal data can be simulated by the transition probability of the regimes of business cycle \citep{Die-Ino}. In addition, it is also found that the transition probability could be identified in non-linear regression approach using Markov chain theory  \citep{hami89,kim-nel}. However, the method for estimating the Markov transition probability remains as the most challenging part, especially with hidden Markov chain models. Attempts to overcome the limitation can be found in a great deal of development on Bayesian estimation and Markov Chain Monte Carlo (MCMC) simulation \citep{wago}.  However, from the standpoint of our research, overcoming the problem of measurement errors and bias from aggregate monitoring data has not been exhaustively achieved. 

Beside the problems concerning the estimation methodology, incomplete monitoring data, and measurement errors, another attractive issue in management of infrastructure is the problems in monitoring and management of underground infrastructure, where monitoring and repair demands a great amount of resources. For example, the leakage of pipeline in a mega city, if happens, will results in a huge social loss and damage other infrastructures. Therefore, it is important to define a best timing for monitoring and repair of underground infrastructure facilities. 

Optimal renewal strategies for underground infrastructure system have long been studied. References could be dated back to 1970s with model of \citet{shamir79} for pipeline network. This model introduced a simple mathematical formulation to estimate the optimal replacement time where the failure rate was assumed to follow the exponential distribution with respect to time. An optimal period for renewal was defined as the period that minimizes the life cycle cost over a certain planning horizon. Other past research similarly employed the analysis of expected life cycle cost in combination with failure rate models subjected to non-homogeneous Poisson process\cite{park00,hong06,Kleiner01}. The rule of replacement is determined by so-called ``critical level'' that is a probability level of failure rate along with time.

A comprehensive study of \citet{jido} further discusses the optimal and repair strategies. In his model, the condition state of infrastructure facility is in continuous state. This assumption purposely encompasses the model for general case as well applicable for various infrastructure structures. The optimal inspection time and repair/replacement condition state are simultaneously solved by using numerical analysis. This model can be applied for underground infrastructure systems; however, the numerical computation of the model remains as an challenge since it requires a high degree of integration.

Along the line of stochastic modeling in infrastructure management, mixture hazard model has been profoundly discussed. Mixture hazard model is used in the case of inhomogeneous monitoring data. However, estimation approach in connection with Markov hazard models has not been studied. Mixture hazard model focuses on estimation of heterogeneity factor of individual group in the same set of monitoring data. If it is possible to develop a methodology for estimation of heterogeneity factor with respect to different groups in one infrastructure system, then study on benchmarking to find out the best infrastructure group for long-term implementation will become feasible.

The mixture hazard model is considered as an excellent technical tool for benchmarking study, which is used to find the best practice in management and technology. The benchmarking study is crucial important in the infrastructure management practices of developing countries. In developing countries, the infrastructure asset management is about at the outset of development. Thus, various problems occur across all facets of existing infrastructure management system. For example, the pavement management system in Vietnam is facing a fast deterioration of its infrastructure\cite{namieee}. However, it seems that the country has not experienced to find out an appropriate management program. Neither effort in the application of HDM4 \footnote{HDM4-Highway Development and Management version 4, developed by the World Bank group, is forced to be applied in recipient country of funds by the World Bank. However, the core part of its model is hidden as a black box} nor ROSY \footnote{A road and pavement management program developed by Carl Bro Pavement Consultants, based in Denmark. The system was applied in Vietnam. However, only the database part is functioning. Other components of ROSY fail to perform their knowledge capabilities} has shown out a positive future direction. It might be due to a poor system database or even due to the operations of the programs themselves since the programs are not yet opened but concealed as the black boxes.
%%%
\section{Objectives of Research}
The objectives for development of this paper are mainly favored by the current spatial distribution of finding problems. In short, it can be categorized into three concrete items as follows:
\begin{itemize}
\item {Developing theoretical deterioration forecasting models with focuses on application of Markov chain process for infrastructure management system. Particularly, focuses are on building model with multi-stage Weibull hazard functions and model on tackling the measurement errors.}
\item {Developing methodologies for application of hazard models in real situation, special attention will be drawn on optimal renewal strategy with regard to technology innovation and benchmarking study with mixture hazard model.} 
\item {Verifying and applying hazard models in the real world by conducting empirical studies on the utilities of tunnel lighting system, road pavement and water supply pipelines.}
\end{itemize}
%%%%%%%%%%%
\section{Scope of Research}
Outlines of scopes are given as follows
\begin{itemize}
\item {Employing Weibull distribution function to solve the problem of selecting optimal inspection time and minimum cumulative cost for repair and renewal of infrastructure utility, empirical study was conducted on utilities of tunnel lighting system in the Japanese expressway. This study is written in chapter \ref{Chapter3}.}
\item {Chapter \ref{Chapter4} discusses the measurement errors in hazard analysis. Based on Bayesian rule and Markov Chain Monte Carlo method, a new type of hazard model is introduced. Empirical test was conducted on database of the highway network in Japan.}
\item {The scope of chapter \ref{Chapter5} is to formulate an optimal time-dependent renewal model for system with difficulty in inspection and monitoring. Empirical study was implemented on underground pipelines system of Osaka city. Particularly, the model of chapter \ref{Chapter5} is developed by in a close link with the content of Chapter \ref{Chapter3}.}
\item {In chapter \ref{Chapter6}, focus is on development of mixture model and benchmarking approach.  Specifically, discussion is on estimation approach for heterogeneity factors and a framework for benchmarking application. Empirical application was analyzed on Vietnamese pavement system.}
\item {Conclusions and recommendations on models and empirical studies are given at every last section of respective chapters.}
\end{itemize}
The link among the topics being presented in four main chapters can be noticed through the definition and description of hazard rate, Weibull or exponential hazard functions and stochastic estimation approach.
%%%%%%%%%%
\section{Expected Contribution}
It is very positively that, after completing the research, the paper and the knowledge of this research will contribute to some extend as follows:
\begin{itemize}
\item {Hazard models in chapter \ref{Chapter3} and chapter \ref{Chapter4} are hoped to bring in innovative academic contributions, particularly about the estimation methodologies. The empirical studies might be extended to cover not only tunnel lighting utilities, pavement system but also other infrastructure system.}
\item {In chapter \ref{Chapter5}, development of a new analytical methodology for optimal renewal is necessary for practical management application of underground system. Especially, empirical test will be carried out on the monitoring data of water pipeline system in Osaka city, estimation results might support Osaka water bureau in choosing the right time and right location for annual renewal activities.}
\item {Infrastructure asset management in developing countries is facing many problems and difficulties. Resources, materials and researches are insufficient. However, the current practices urgently demand an effective analytical and systemic approach to cope with high rate of deterioration. It is therefore, the study of chapter \ref{Chapter6} hopes to bring in a new contribution, by some means or other, to facilitate the development of infrastructure asset management in developing countries.}
\end{itemize}

